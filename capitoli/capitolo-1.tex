% !TEX encoding = UTF-8
% !TEX TS-program = pdflatex
% !TEX root = ../tesi.tex

%**************************************************************
\chapter{Contesto aziendale}
\label{cap:introduzione}
%**************************************************************

%**************************************************************
\section{Lab Network}

\begin{figure}[h]
	\begin{center}
	\includegraphics[scale=0.4]{immagini/LOGO_LABNETWORK.png}
	\caption{Logo \lab{}}
	\end{center}
\end{figure}

\lab{} nasce nel XXXX con lo scopo di aiutare le imprese ad innovare prodotti e processi attraverso la competenza concreta dei laboratori, sfruttando le potenzialità dei moderni strumenti digitali.

\section{Organizzazione aziendale}
Descrizione dei processi aziendali, dell'organigramma e della metodologia di lavoro che l'azienda ha adottato e affinato negli anni.


\section{Prodotti e servizi}
Descrizione dei prodotti e dei servizi offerti dall'azienda.
\subsubsection{Prodotti}

\subsubsection{Servizi}
I servizi che \lab{} offre ai propri clienti sono:
\begin{itemize}
\item \textbf{Progettazione:} corsi di formazione su misura, \gls{counseling} e \gls{workshop} per imparare a sfruttare in modo professionale: 3D Printing, schede elettroniche, \gls{cz}, realtà virtuale e aumentata, sviluppo App, prototipazione, \gls{bigd}, \gls{iot} ecc.
\item \textbf{Noleggio:} Kit e attrezzature come stampanti 3D, schede elettroniche (\gls{arduino}, \gls{rpi}, Intel) visori VR, pc e notebook, videoproiettori; affitto aule didattiche per \gls{makers} e industria 4.0.
\item \textbf{Personale qualificato:}
\item \textbf{Assistenza:}
\end{itemize}

\section{ln Network e innovazione}
In questa sezione descriverò il rapporto che \lab{} ha con l'innovazione, ovvero come investe su di essa in termini di prodotti, servizi e personale.