% !TEX encoding = UTF-8
% !TEX TS-program = pdflatex
% !TEX root = ../tesi.tex

%**************************************************************
\chapter{Valutazione retrospettiva}
\label{cap:valutazione-retrospettiva}
%**************************************************************

\section{Obiettivi prefissati}
%Breve bilancio sugli obiettivi raggiunti in rapporto a quelli preventivati.
Al termine del progetto di stage ho analizzato gli obiettivi raggiunti rispetto a quelli fissati inizialmente. Di seguito riporto una tabella con le valutazioni.

\renewcommand{\arraystretch}{1.4}
\begin{longtable}{|p{7cm}|p{2.5cm}|p{3cm}|}
\hline
\textbf{Obiettivo} & \textbf{Tipo} & \textbf{Esito finale} \\ 
\hline
\textbf{Integrazione di un sistema completo per l'apertura di serrature con lettura di codice a barre e NFC} & Obbligatorio & Soddisfatto \\ 
\hline
\textbf{Realizzazione della piattaforma web per la gestione degli accessi} & Obbligatorio & Soddisfatto \\ 
\hline
\textbf{Creazione del modello 3D dell'involucro e sua realizzazione con stampa 3D} & Obbligatorio & Soddisfatto \\ 
\hline
\textbf{Redazione della manualistica completa} & Obbligatorio & Soddisfatto \\ 
\hline
\textbf{Cura e definizione dell'interfaccia grafica della piattaforma web} & Desiderabile & Soddisfatto \\ 
\hline
\textbf{Ottimizzazione del sistema esistente in termini di efficienza e prestazioni} & Desiderabile & Non soddisfatto \\ 
\hline
\textbf{Creazione di un modello 3D modulare espandibile per future versioni} & Facoltativo & Non soddisfatto \\ 
\hline
\caption{Valutazione finale sugli obiettivi aziendali.}
\end{longtable}

\begin{itemize}
\item Mancanza di tempo
\item Conoscenza limitata o assente
\item Impegni aziendali
\item Alternanza lavorativa
\end{itemize}

\section{Obiettivi personali e valutazione formativa}
Al termine dello stage devo ritenermi soddisfatto rispetto alle aspettative iniziali di crescita personale. Questo progetto di stage mi ha permesso di mettere in pratica buona parte delle conoscenze acquisite durante il mio percorso di studi e di arricchire la mia formazione con nuove tecnologie.

\begin{itemize}
\item \textbf{Interazione tra hardware e software}: 
\item \textbf{Modellazione e stampa 3D}:
\item \textbf{Sviluppo software in team}:
\end{itemize}

\section{Distanza tra università e lavoro}
Analisi sulla distanza (o vicinanza) tra il corso di studi (e quindi le conoscenze apprese durante lo studio) e il mondo lavorativo incontrato durante lo stage, mettendo in evidenza le lacune che si sono dovute colmare per completare il tirocinio.
Consigli al corso di studi sulla base dell'esperienza personale.