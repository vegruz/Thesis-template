% !TEX encoding = UTF-8
% !TEX TS-program = pdflatex
% !TEX root = ../tesi.tex

%**************************************************************
\chapter{Valutazione retrospettiva}
\label{cap:valutazione-retrospettiva}
%**************************************************************

\section{Obiettivi prefissati}
%Breve bilancio sugli obiettivi raggiunti in rapporto a quelli preventivati.
Al termine del progetto di stage ho analizzato gli obiettivi raggiunti rispetto a quelli fissati inizialmente. Di seguito riporto una tabella con le valutazioni.

\renewcommand{\arraystretch}{1.4}
\begin{longtable}{|p{7cm}|p{2.5cm}|p{3cm}|}
\hline
\textbf{Obiettivo} & \textbf{Tipo} & \textbf{Esito finale} \\ 
\hline
\textbf{Integrazione di un sistema completo per l'apertura di serrature con lettura di codice a barre e NFC} & Obbligatorio & Soddisfatto \\ 
\hline
\textbf{Realizzazione della piattaforma web per la gestione degli accessi} & Obbligatorio & Soddisfatto \\ 
\hline
\textbf{Creazione del modello 3D dell'involucro e sua realizzazione con stampa 3D} & Obbligatorio & Soddisfatto \\ 
\hline
\textbf{Redazione della manualistica completa} & Obbligatorio & Soddisfatto \\ 
\hline
\textbf{Cura e definizione dell'interfaccia grafica della piattaforma web} & Desiderabile & Soddisfatto \\ 
\hline
\textbf{Ottimizzazione del sistema esistente in termini di efficienza e prestazioni} & Desiderabile & Non soddisfatto \\ 
\hline
\textbf{Creazione di un modello 3D modulare espandibile per future versioni} & Facoltativo & Non soddisfatto \\ 
\hline
\caption{Valutazione finale sugli obiettivi aziendali.}
\end{longtable}

Sono alcune le motivazioni per cui alcuni degli obiettivi prefissati non sono stati raggiunti:

\begin{itemize}
\item \textbf{Conoscenza limitata o assente}: per lo svolgimento del progetto di stage ho lavorato con nuove tecnologie e strumenti mai utilizzati prima d'ora. La fase di studio iniziale è stata molto utile per l'apprendimento sul funzionamento di Arduino e la sua programmazione, al contrario, il periodo di formazione personale precedente allo sviluppo, non è stato sufficiente per un apprendimento esaustivo sul framework Laravel. Questo ha comportato delle piccole attività di ricerca anche durante la fase di codifica rallentandola.

Analogamente, la conoscenza basilare del software Rhinoceros mi ha permesso di sviluppare solamente un modello base non espandibile per eventuali versioni future.
\item \textbf{Alternanza lavorativa}: essendo studente lavoratore, ho dovuto trovare un compromesso con il tutor aziendale per lo svolgimento dello stage: ho svolto le prime cinque settimane a tempo pieno fino al raggiungimento di 200 ore, mentre le rimanenti ore sono state suddivise su una base di 24 ore settimanali. Questa suddivisione ha compromesso la fase di codifica, in quanto non ho potuto dare continuità al lavoro di sviluppo, rallentando, seppur in percentuale minima, la realizzazione del progetto.
\item \textbf{Mancanza di tempo}: a causa delle motivazioni sopra descritte, la fase di codifica ha richiesto un tempo maggiore a quanto pianificato e pertanto il tempo a disposizione non è stato sufficiente per il raggiungimento di alcuni obiettivi.
\end{itemize}

\section{Obiettivi personali e valutazione formativa}
Al termine dello stage devo ritenermi soddisfatto rispetto alle aspettative iniziali di crescita personale. Questo progetto di stage mi ha permesso di mettere in pratica buona parte delle conoscenze acquisite durante il mio percorso di studi e di arricchire la mia formazione con nuove tecnologie.

\begin{itemize}
\item \textbf{Interazione tra hardware e software}: prima dell'inizio dello stage era mio interesse imparare la programmazione hardware e comprendere come il software interagisce con le componenti elettroniche. A tirocinio terminato sono soddisfatto rispetto a quelle che erano le mie aspettative su questo punto: ho appreso il funzionamento di Arduino, strumento open source ormai largamente diffuso e utilizzato per la prototipazione rapida, la sua programmazione tramite l'IDE dedicato e alcune basi di elettronica.
\item \textbf{Modellazione e stampa 3D}: lo svolgimento di questo progetto mi ha permesso di conoscere il mondo della stampa 3D molto da vicino. A stage terminato ho acquisito delle utili competenze che variano dalla modellazione 3D alla stampa del modello.
\item \textbf{Sviluppo software in team}: durante il percorso universitario ho svolto progetti didattici, come ad esempio durante il corso di Ingegneria del Software, mirati a simulare nel modo più fedele possibile lo sviluppo di un software nel mondo del lavoro. Grazie allo stage, ho avuto la possibilità di confrontare quanto vissuto nei progetti didattici con le problematiche reali di un progetto lavorativo. Questo implica lavorare con un team di sviluppo eterogeneo, composto da sviluppatori con competenze e abilità a volte molto diverse.
\end{itemize}

\section{Università e mondo del lavoro}
Il compito dell'Università non è limitato alla formazione di uno studente su vari fronti, ma si estende alla sua preparazione per affrontare, al termine del percorso di studio, il suo inserimento al mondo del lavoro. Questo non significa che uno studente laureato debba essere già in grado di svolgere un lavoro, ma deve essere preparato ad affrontare le problematiche che il mondo del lavoro presenta. 

\medskip

\textbf{Preparazione generale}: sulla base della mia esperienza, uno studente universitario grazie alla preparazione ricevuta, è in grado di saper gestire in autonomia il lavoro assegnatogli sapendo affrontare anche situazioni impreviste. Inoltre possiede un'ottima capacità di adattamento rispetto allo studio di nuove tecnologie e all'utilizzo di nuovi strumenti di lavoro.

\medskip

\textbf{Nell'ambito informatico}: la preparazione complessiva è varia e abbastanza completa, questo permette l'inserimento in diverse realtà lavorative. Dalla mia esperienza posso concludere che sono state presenti alcune lacune derivate da....
