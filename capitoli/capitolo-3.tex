% !TEX encoding = UTF-8
% !TEX TS-program = pdflatex
% !TEX root = ../tesi.tex

%**************************************************************
\chapter{Lo stage}
\label{cap:descrizione-stage}
%**************************************************************

\section{Pianificazione}
Anticipatamente all'inizio del progetto, ho redatto assieme al tutor aziendale un piano di lavoro, riportante la pianificazione sia degli obiettivi che della suddivisione temporale.

All'interno di questo documento ho riportato la ripartizione delle ore rispetto alle varie attività.

\bigskip

% !TEX encoding = UTF-8
% !TEX TS-program = pdflatex
% !TEX root = ../tesi.tex

% Tabella da personalizzare in base alle ore delle attività

\begin{tabularx}{\textwidth}{|c|X|}
	\hline
	\textbf{Durata in ore} & \textbf{Descrizione dell'attività} \\\hline
	
	\textbf{48} & \textbf{Formazione sulle tecnologie utilizzate} \\	 
    \hline
    
    \textbf{72} & \textbf{Definizione del sistema hardware/software e relativa documentazione} \\ \hdashline 
    \multirow{3}{0cm}\\ 
    \textit{16} & 
    \textit{Analisi del problema e del dominio applicativo} \\
    \textit{26} & 
    \textit{Adattamento e revisione della piattaforma esistente} \\
    \textit{8} & 
    \textit{Test piattaforma hardware} \\
    \textit{18} & 
    \textit{Progettazione e sviluppo software CRM} \\
    \textit{4} & 
    \textit{Stesura documentazione} \\
    \hline
    
    \textbf{24} & \textbf{Modellazione e Stampa 3D}  \\ \hdashline 
    \multirow{4}{0cm}\\ 
    \textit{18} & 
    \textit{Design e modellazione involucro} \\
    \textit{6} & 
    \textit{Slicing del modello e stampa 3D} \\
    \hline
    
    \textbf{48} & \textbf{Sviluppo piattaforma online}  \\ \hdashline 
    \multirow{4}{0cm}\\ 
    \textit{18} & 
    \textit{Pianificazione e analisi dei requisiti} \\
    \textit{30} & 
    \textit{Sviluppo della piattaforma con iniziale attenzione al lato server} \\
    \hline
    
    \textbf{40} & \textbf{Cura dell'interfaccia grafica}  \\ 
    \hline
    
    \textbf{16} & \textbf{Test e verifica presenza bug}  \\
    \hline
    
	\textbf{48} & \textbf{Redazione dei manuali d'uso}  \\
    \hline    
    
    \textbf{12} & \textbf{Collaudo Finale}  \\ \hdashline 
    \multirow{4}{0cm}\\ 
    \textit{8} & 
    \textit{Collaudo} \\
    \textit{2} & 
    \textit{Incontro di presentazione della piattaforma con gli stakeholders} \\
    \textit{2} & 
    \textit{Live demo di tutto il lavoro di stage} \\
    \hline
	
	\textbf{Totale ore} & \multicolumn{1}{|c|}{\textbf{\totaleOre}} \\\hline
	
	
\end{tabularx}

%**************************************************************
\section{Studio del sistema}
%Il progetto verrà spiegato in questa sezione, senza però scendere in dettagli, evitando quindi riproducibilità.
Lo sviluppo del progetto è partito da uno studio preliminare sul sistema già esistente composto da una centralina di controllo ed un lettore di schede NFC.
Scendendo nei dettagli, la fase di studio è iniziata dalla centralina, composta da una scheda Arduino. Per poterne apprendere il funzionamento, il tutor aziendale mi ha fornito una scheda di test sulla quale eseguire del codice di esempio con dei semplici circuiti elettronici. L'apprendimento si è rivelato piuttosto rapido anche grazie alla vasta disponibilità di materiale online e ad una consolidata community.

Una volta appreso il funzionamento generale di Arduino, ho iniziato lo studio sul codice che governa il comportamento della centralina di controllo, caratterizzato principalmente da 3 fasi:

\begin{itemize}
\item \textbf{Configurazione iniziale}: il sistema esegue una configurazione iniziale in modo da integrarsi perfettamente nella rete alla quale è collegato;
\item \textbf{Listener}: una porzione di codice che rimane in continuo ascolto per catturare gli eventi derivanti dai moduli collegati;
\item \textbf{Crittografia e HTTP request}: una funzione che si occupa della cifratura dei codici letti e del loro invio tramite richieste HTTP ad un server.
\end{itemize}

\section{Analisi dei requisiti}
\subsection{Identificazione}
Per poter identificare correttamente tutti i requisiti, a seguito di vari colloqui con il tutor aziendale e il cliente, ho definito i vari casi d'uso, disegnandone in seguito i rispettivi diagrammi UML. Ogni riunione è stata riportata su di un verbale e da quest'ultimo ho in seguito ricavato i vari requisiti. In modo da poter identificare ulteriori requisiti ho analizzato approfonditamente i casi d'uso. Una volta individuati tutti i requisiti sono stati riportati in un documento soggetto a versionamento.

\subsection{Strumenti a supporto}
\subsection{Formalizzazione dei requisiti}

\section{Progettazione}

\section{Codifica}

\section{Verifica e validazione}

\section{Visione generale del progetto}