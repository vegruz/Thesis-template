% !TEX encoding = UTF-8
% !TEX TS-program = pdflatex
% !TEX root = ../tesi.tex

%**************************************************************
\chapter{Il progetto nella strategia aziendale}
\label{cap:processi-metodologie}
%**************************************************************

\section{Azienda e stage}
Lab Network Srl è solita ospitare tirocinanti e stagisti per dare loro una formazione e valutarne le capacità per un possibile inserimento in azienda.

Il tirocinio porta un vantaggio sia per gli stagisti stessi, che hanno modo di mettere in pratica in campo lavorativo quanto appreso durante gli studi, sia per l'azienda che ha modo di valutare le capacità di una persona al fine dell'assunzione formandola al tempo stesso. Gli stage offrono inoltre la possibilità all'azienda di avviare progetti che normalmente non avrebbero spazio. 

Sono venuto a conoscenza di Lab Network grazie alla diffusione mediatica di alcuni loro progetti che stavano avendo successo, come ad esempio Vitruvian Game. Ho deciso quindi di propormi all'azienda per lo svolgimento del tirocinio; Lab Network ha accettato proponendomi alcuni progetti disponibili tra cui FabKey.

Lab Network ha deciso di portare avanti un progetto come FabKey attraverso uno stage perché fino ad allora era stato catalogato come progetto secondario e non aveva ancora trovato spazio in azienda sfruttando il personale a disposizione.


\section{Obiettivi personali}
\begin{itemize}
\item \textbf{Codifica e correlazione tra hardware e software}: un aspetto dell'informatica che mi ha da sempre affascinato è la relazione tra hardware e software; non avendo potuto approfondire l'argomento durante gli studi, ho trovato questa proposta di stage un'ottima opportunità per studiare sul campo la programmazione hardware. Il progetto, inoltre, si affaccia al mondo dell'IOT che è in continua crescita;
\item \textbf{Modellazione e stampa 3D}: un campo ormai molto diffuso e in continua crescita è quello della stampa 3D; con questo progetto di stage mi è stata offerta la possibilità di usare in esclusiva una stampante 3D per la realizzazione dei primi prototipi dopo averli correttamente modellati;
\item Lavoro in team
\end{itemize}

Lavoro in team in azienda sw e conoscere il mondo del lavoro in ambito sw

\section{Obiettivi aziendali}
1.	Obbligatori

O01: Integrazione di un sistema completo per l'apertura di serrature con lettura di codice a barre e NFC; 

O02: Realizzazione della piattaforma web per la gestione degli accessi; 

O03: Creazione del modello 3D dell'involucro e sua realizzazione con stampa 3D; 

O04: Redazione della manualistica completa; 

Desiderabili

D01: Cura e definizione dell'interfaccia grafica della piattaforma web;

D02: Ottimizzazione del sistema esistente in termini di efficienza e prestazioni; 


Facoltativi 

F01: Creazione di un modello 3D modulare espandibile per future versioni; 


\section{Vincoli}
Vincoli che l'azienda ha imposto nei miei confronti durante lo svolgimento dello stage (ad esempio: vincoli di tempo, tecnologici e di metodo).