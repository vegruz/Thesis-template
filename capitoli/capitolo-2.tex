% !TEX encoding = UTF-8
% !TEX TS-program = pdflatex
% !TEX root = ../tesi.tex

%**************************************************************
\chapter{Il progetto nella strategia aziendale}
\label{cap:processi-metodologie}
%**************************************************************

\section{Azienda e stage}
Lab Network Srl è solita ospitare tirocinanti e stagisti per dare loro una formazione e valutarne le capacità per un possibile inserimento in azienda.

Il tirocinio porta un vantaggio sia per gli stagisti stessi, che hanno modo di mettere in pratica in campo lavorativo quanto appreso durante gli studi, sia per l'azienda che ha modo di valutare le capacità di una persona al fine dell'assunzione formandola al tempo stesso. Gli stage offrono inoltre la possibilità all'azienda di avviare progetti che normalmente non avrebbero spazio. 

Sono venuto a conoscenza di Lab Network grazie alla diffusione mediatica di alcuni loro progetti che stavano avendo successo, come ad esempio Vitruvian Game. Ho deciso quindi di propormi all'azienda per lo svolgimento del tirocinio; Lab Network ha accettato proponendomi alcuni progetti disponibili tra cui FabKey.

Lab Network ha deciso di portare avanti un progetto come FabKey attraverso uno stage perché fino ad allora era stato catalogato come progetto secondario e non aveva ancora trovato spazio in azienda sfruttando il personale a disposizione.


\section{Obiettivi personali}
\begin{itemize}
\item \textbf{Codifica e correlazione tra hardware e software}: un aspetto dell'informatica che mi ha da sempre affascinato è la relazione tra hardware e software; non avendo potuto approfondire l'argomento durante gli studi, ho trovato questa proposta di stage un'ottima opportunità per studiare sul campo la programmazione hardware. Il progetto, inoltre, si affaccia al mondo dell'IOT che è in continua crescita;
\item \textbf{Modellazione e stampa 3D}: un campo ormai molto diffuso e in continua crescita è quello della stampa 3D; con questo progetto di stage mi è stata offerta la possibilità di usare in esclusiva una stampante 3D per la realizzazione dei primi prototipi dopo averli correttamente modellati;
\item \textbf{Lavoro in team}: fare parte di un team di sviluppo cogliendo gli aspetti in comune e le differenze tra un progetto accademico come quello svolto durante il corso di Ingegneria del Software e un progetto in ambito lavorativo.
\end{itemize}

Dallo svolgimento dello stage presso \lab{} mi aspetto di:
\begin{itemize}
\item Imparare la programmazione di schede open source come Arduino affacciandomi quindi al mondo dell'IOT in forte espansione;
\item Capire a fondo il principio di funzionamento della stampa 3D, conoscere le varie problematiche e relative soluzioni. Parallelamente apprendere le basi per la modellazione 3D
\end{itemize}

\section{Obiettivi aziendali}
L'obiettivo del progetto di stage è l'ampliamento del già collaudato sistema ``FabKey'', il quale permette l'apertura di una porta attraverso un tag NFC controllando una lista di accessi presente in un database online. Nello specifico, tale sistema deve essere revisionato e ampliato, utilizzando un modulo che andrà ad autenticare l'utente tramite codice a barre anziché tag NFC.

Gli obiettivi concordati con il tutor aziendale sono stati classificati secondo tre gradi di priorità:

\begin{enumerate}
\item \textbf{Obbligatori}: obiettivi il cui sviluppo è necessario per la riuscita del progetto;
\begin{itemize}
\item Integrazione di un sistema completo per l'apertura di serrature con lettura di codice a barre e NFC; 
\item Realizzazione della piattaforma web per la gestione degli accessi; 
\item Creazione del modello 3D dell'involucro e sua realizzazione con stampa 3D; 
\item Redazione della manualistica completa; 
\end{itemize}
\item \textbf{Desiderabili}: il loro sviluppo non è necessario ai fini del progetto, ma forniscono un valore aggiunto considerevole;
\begin{itemize}
\item Cura e definizione dell'interfaccia grafica della piattaforma web;
\item Ottimizzazione del sistema esistente in termini di efficienza e prestazioni; 
\end{itemize}
\item \textbf{Facoltativi}: il loro sviluppo diventa apprezzabile, ma dal valore aggiunto trascurabile.
\begin{itemize}
\item Creazione di un modello 3D modulare espandibile per future versioni; 
\end{itemize}
\end{enumerate}

\section{Vincoli}
Mi sono stati imposti dei vincoli da parte dell'azienda in termini di tempo e tecnologie al fine di un corretto svolgimento del progetto oltre che una buona integrazione con il resto del team di sviluppo.

I vincoli tecnologici sono:
\begin{itemize}
\item GitLab (gestione ticket, product Backlog, versionamento...)
\item Arduino IDE
\item Rhinoceros
\end{itemize}
