% !TEX encoding = UTF-8
% !TEX TS-program = pdflatex
% !TEX root = ../tesi.tex

%**************************************************************
% Sommario
%**************************************************************
\cleardoublepage
\phantomsection
\pdfbookmark{Sommario}{Sommario}
\begingroup
\let\clearpage\relax
\let\cleardoublepage\relax
\let\cleardoublepage\relax

\chapter*{Sommario}

Il presente documento descrive il lavoro svolto durante il periodo di stage, della durata di circa trecento ore, dal laureando Federico Vegro presso l'azienda Lab Network S.r.l.
Lo scopo principale del progetto di stage era l'ampliamento del già collaudato sistema ``FabKey'', il quale permette l'apertura di una porta attraverso un tag NFC controllando una lista di accessi presente in un database online. Nello specifico, tale sistema è stato revisionato e ampliato, utilizzando un modulo che permette l'autenticazione dell'utente tramite codice a barre anziché tag NFC.

Gli obbiettivi da raggiungere erano molteplici sono stati:
\begin{itemize}
\item curare l'interfaccia web che consente la gestione dei permessi da parte dell'amministratore;
\item creare una pagina web che permetta l'inserimento di nuovi codici di accesso autorizzati;
\item revisionare ed adattare il codice per interfacciare il nuovo modulo al sistema già presente;
\item curare la parte estetica del modulo di lettura avvalendosi di software per la modellazione 3D e successivamente alla stampa 3D;
\item occuparsi della stesura della documentazione necessaria al corretto utilizzo della piattaforma.
\end{itemize}

\bigskip

\textbf{Convenzioni tipografiche}

\bigskip

Per la stesura del presente documento sono state adottate alcune convenzioni tipografiche al fine di migliorarne la leggibilità:
\begin{itemize}
\item alla fine del documento è presente un glossario, nel quale vengono definiti acronimi, abbreviazioni e termini di uso non comune. 

La prima ricorrenza dei termini è contrassegnata nel testo con la seguente nomenclatura: \textit{termine}\textbf{\textsubscript{[G]}};
\item i termini in lingua straniera sono stati formattati in \textit{corsivo};
\item viene posto un riferimento numerico del tipo \textbf{[01]} per indicare i riferimenti bibliografici.
\end{itemize}

%\vfill
%
%\selectlanguage{english}
%\pdfbookmark{Abstract}{Abstract}
%\chapter*{Abstract}
%
%\selectlanguage{italian}

\endgroup			

\vfill

