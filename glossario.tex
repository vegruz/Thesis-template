%**************************************************************
% Glossario
%**************************************************************
\renewcommand{\glossaryname}{Glossario}

\newglossaryentry{api}
{
    name=\glslink{api}{API},
    text=API,
    sort=api,
    description={in informatica con il termine \emph{Application Programming Interface API} (ing. interfaccia di programmazione di un'applicazione) si indica ogni insieme di procedure disponibili al programmatore, di solito raggruppate a formare un set di strumenti specifici per l'espletamento di un determinato compito all'interno di un certo programma. La finalità è ottenere un'astrazione, di solito tra l'hardware e il programmatore o tra software a basso e quello ad alto livello semplificando così il lavoro di programmazione [08]}
}

\newglossaryentry{uml}
{
    name=\glslink{uml}{UML},
    text=UML,
    sort=uml,
    description={in ingegneria del software \emph{UML, Unified Modeling Language} (ing. linguaggio di modellazione unificato) è un linguaggio di modellazione e specifica basato sul paradigma object-oriented. L'\emph{UML} svolge un'importantissima funzione di ``lingua franca'' nella comunità della progettazione e programmazione a oggetti. Gran parte della letteratura di settore usa tale linguaggio per descrivere soluzioni analitiche e progettuali in modo sintetico e comprensibile a un vasto pubblico [09]}
}

\newglossaryentry{fabkey}
{
    name=\glslink{fabkey}{FabKey},
    text=FabKey,
    sort=fabkey,
    description={Chiave IoT online e \textit{open source}, oggetto del progetto di stage, successivamente rinominata in LabKey}
}

\newglossaryentry{counseling}
{
	name=\glslink{counseling}{Counseling},
    text=counseling,
    sort=counseling,
    description={indica un'attività professionale che tende ad orientare, sostenere e sviluppare le potenzialità del soggetto, promuovendone atteggiamenti attivi, propositivi e stimolando le capacità di scelta [10]}
}

\newglossaryentry{workshop}
{
	name=\glslink{workshop}{Workshop},
    text=workshop,
    sort=workshop,
    description={letteralmente, dall'inglese, laboratorio. Il termine viene utilizzato per indicare incontri e riunioni in cui tutti i partecipanti sono protagonisti attivi, animano la discussione, condividono idee ed elaborano soluzioni, raggiungono risultati tangibili [11]}
}

\newglossaryentry{bigd}
{
	name=\glslink{bigd}{Big Data},
    text=Big Data,
    sort=big,
    description={il termine descrive l'insieme delle tecnologie e delle metodologie di analisi di dati massivi, ovvero la capacità di estrapolare, analizzare e mettere in relazione un'enorme mole di dati eterogenei, strutturati e non strutturati, per scoprire i legami tra fenomeni diversi e prevedere quelli futuri [12]}
}

\newglossaryentry{iot}
{
	name=\glslink{iot}{IoT},
    text=IoT,
    sort=iot,
    description={acronimo dell'inglese \textit{Internet Of Things}, letteralmente \textit{Internet degli oggetti}; è un neologismo riferito all'estensione di Internet al mondo degli oggetti e dei luoghi concreti [13]}
}

\newglossaryentry{arduino}
{
	name=\glslink{arduino}{Arduino},
    text=Arduino,
    sort=arduino,
    description={si tratta di una piattaforma hardware composta da una serie di schede elettroniche dotate di un microcontrollore. È abbinato ad un semplice ambiente di sviluppo integrato per la programmazione del microcontrollore. Tutto il software a corredo è libero, e gli schemi circuitali sono distribuiti come hardware libero [14]}
}

\newglossaryentry{rpi}
{
	name=\glslink{rpi}{Raspberry Pi},
    text=Raspberry Pi,
    sort=raspberry,
    description={è un \textit{single-board computer}, ovvero una scheda elettronica implementante un intero computer o quasi [15]}
}

\newglossaryentry{FabLab}
{
	name=\glslink{FabLab}{FabLab},
    text=FabLab,
    sort=fablab,
    description={dall'inglese \textit{fabrication laboratory}, è una piccola officina che offre servizi personalizzati di fabbricazione digitale. Un fab lab è generalmente dotato di una serie di strumenti computerizzati in grado di realizzare, in maniera flessibile e semi-automatica, un'ampia gamma di oggetti [16]}
}

\newglossaryentry{NFC}
{
	name=\glslink{NFC}{NFC},
    text=NFC,
    sort=nfc,
    description={acronimo dell'inglese \textit{Near Field Communication}, ossia comunicazione di prossimità, è una tecnologia che fornisce connettività senza fili (RF) bidirezionale a corto raggio (fino a un massimo di 10 cm) [17]}
}

\newglossaryentry{SWOT}
{
	name=\glslink{SWOT}{SWOT, analisi},
    text=analisi SWOT,
    sort=SWOT,
    description={è uno strumento di pianificazione strategica usato per valutare i punti di forza (\textit{Strengths}), le debolezze (\textit{Weaknesses}), le opportunità (\textit{Opportunities}) e le minacce (\textit{Threats}) di un progetto o in un'impresa o in ogni altra situazione in cui un'organizzazione o un individuo debba svolgere una decisione per il raggiungimento di un obiettivo [18]}
}


%\newglossaryentry{}
%{
%	name=\glslink{}{},
%    text=,
%    sort=,
%    description={}
%}
