%**************************************************************
% file contenente le impostazioni della tesi
%**************************************************************

%**************************************************************
% Frontespizio
%**************************************************************

% Autore
\newcommand{\myName}{Federico Vegro}                                    
\newcommand{\myTitle}{FabKey: sviluppo della serratura online, tra IoT e automazione}

% Tipo di tesi                   
\newcommand{\myDegree}{Tesi di laurea triennale}

% Università             
\newcommand{\myUni}{Università degli Studi di Padova}

% Facoltà       
\newcommand{\myFaculty}{Corso di Laurea in Informatica}

% Dipartimento
\newcommand{\myDepartment}{Dipartimento di Matematica "Tullio Levi-Civita"}

% Titolo del relatore
\newcommand{\profTitle}{Prof.}

% Relatore
\newcommand{\myProf}{Tullio Vardanega}

% Luogo
\newcommand{\myLocation}{Padova}

% Anno accademico
\newcommand{\myAA}{2017-2018}

% Data discussione
\newcommand{\myTime}{Dicembre 2018}

% Azienda
\newcommand{\lab}{Lab Network Srl}


%**************************************************************
% Impostazioni di impaginazione
% see: http://wwwcdf.pd.infn.it/AppuntiLinux/a2547.htm
%**************************************************************

\setlength{\parindent}{0pt}   % larghezza rientro della prima riga
\setlength{\parskip}{0pt}   % distanza tra i paragrafi


%**************************************************************
% Impostazioni di biblatex
%**************************************************************
\bibliography{bibliografia} % database di biblatex 

\defbibheading{bibliography} {
    \cleardoublepage
    \phantomsection 
    \addcontentsline{toc}{chapter}{\bibname}
    \chapter*{\bibname\markboth{\bibname}{\bibname}}
}

\setlength\bibitemsep{1.5\itemsep} % spazio tra entry

\DeclareBibliographyCategory{opere}
\DeclareBibliographyCategory{web}

\addtocategory{opere}{womak:lean-thinking}
\addtocategory{web}{site:agile-manifesto}

\defbibheading{opere}{\section*{Riferimenti bibliografici}}
\defbibheading{web}{\section*{Siti Web consultati}}


%**************************************************************
% Impostazioni di caption
%**************************************************************
\captionsetup{
    tableposition=top,
    figureposition=bottom,
    font=small,
    format=hang,
    labelfont=bf
}

%**************************************************************
% Impostazioni di glossaries
%**************************************************************
%**************************************************************
% Glossario
%**************************************************************
\renewcommand{\glossaryname}{Glossario}

\newglossaryentry{api}
{
    name=\glslink{api}{API},
    text=API,
    sort=api,
    description={in informatica con il termine \emph{Application Programming Interface API} (ing. interfaccia di programmazione di un'applicazione) si indica ogni insieme di procedure disponibili al programmatore, di solito raggruppate a formare un set di strumenti specifici per l'espletamento di un determinato compito all'interno di un certo programma. La finalità è ottenere un'astrazione, di solito tra l'hardware e il programmatore o tra software a basso e quello ad alto livello semplificando così il lavoro di programmazione [8]}
}

\newglossaryentry{uml}
{
    name=\glslink{uml}{UML},
    text=UML,
    sort=uml,
    description={in ingegneria del software \emph{UML, Unified Modeling Language} (ing. linguaggio di modellazione unificato) è un linguaggio di modellazione e specifica basato sul paradigma object-oriented. L'\emph{UML} svolge un'importantissima funzione di ``lingua franca'' nella comunità della progettazione e programmazione a oggetti. Gran parte della letteratura di settore usa tale linguaggio per descrivere soluzioni analitiche e progettuali in modo sintetico e comprensibile a un vasto pubblico [9]}
}

\newglossaryentry{fabkey}
{
    name=\glslink{fabkey}{FabKey},
    text=FabKey,
    sort=fabkey,
    description={Chiave IoT online e \textit{open source}, oggetto del progetto di stage, successivamente rinominata in LabKey}
}

\newglossaryentry{counseling}
{
	name=\glslink{counseling}{Counseling},
    text=counseling,
    sort=counseling,
    description={indica un'attività professionale che tende ad orientare, sostenere e sviluppare le potenzialità del soggetto, promuovendone atteggiamenti attivi, propositivi e stimolando le capacità di scelta [10]}
}

\newglossaryentry{workshop}
{
	name=\glslink{workshop}{Workshop},
    text=workshop,
    sort=workshop,
    description={letteralmente, dall'inglese, laboratorio. Il termine viene utilizzato per indicare incontri e riunioni in cui tutti i partecipanti sono protagonisti attivi, animano la discussione, condividono idee ed elaborano soluzioni, raggiungono risultati tangibili [11]}
}

\newglossaryentry{bigd}
{
	name=\glslink{bigd}{Big Data},
    text=Big Data,
    sort=big,
    description={il termine descrive l'insieme delle tecnologie e delle metodologie di analisi di dati massivi, ovvero la capacità di estrapolare, analizzare e mettere in relazione un'enorme mole di dati eterogenei, strutturati e non strutturati, per scoprire i legami tra fenomeni diversi e prevedere quelli futuri [12]}
}

\newglossaryentry{iot}
{
	name=\glslink{iot}{IoT},
    text=IoT,
    sort=iot,
    description={acronimo dell'inglese \textit{Internet Of Things}, letteralmente \textit{Internet degli oggetti}; è un neologismo riferito all'estensione di Internet al mondo degli oggetti e dei luoghi concreti [13]}
}

\newglossaryentry{arduino}
{
	name=\glslink{arduino}{Arduino},
    text=Arduino,
    sort=arduino,
    description={si tratta di una piattaforma hardware composta da una serie di schede elettroniche dotate di un microcontrollore. È abbinato ad un semplice ambiente di sviluppo integrato per la programmazione del microcontrollore. Tutto il software a corredo è libero, e gli schemi circuitali sono distribuiti come hardware libero [14]}
}

\newglossaryentry{rpi}
{
	name=\glslink{rpi}{Raspberry Pi},
    text=Raspberry Pi,
    sort=raspberry,
    description={è un \textit{single-board computer}, ovvero una scheda elettronica implementante un intero computer o quasi [15]}
}

\newglossaryentry{makers}
{
	name=\glslink{makers}{Makers},
    text=makers,
    sort=makers,
    description={il termine identifica gli artigiani digitali, i quali costituiscono un movimento culturale contemporaneo che rappresenta un'estensione su base tecnologica del tradizionale mondo del bricolage. Tra gli interessi tipici degli artigiani digitali vi sono realizzazioni di tipo ingegneristico, come apparecchiature elettroniche, realizzazioni robotiche, dispositivi per la stampa 3D, e apparecchiature a controllo numerico. Sono anche contemplate attività più convenzionali, come la lavorazione dei metalli, del legno e l'artigianato tradizionale [16]}
}

\newglossaryentry{FabLab}
{
	name=\glslink{FabLab}{FabLab},
    text=FabLab,
    sort=fablab,
    description={dall'inglese \textit{fabrication laboratory}, è una piccola officina che offre servizi personalizzati di fabbricazione digitale. Un fab lab è generalmente dotato di una serie di strumenti computerizzati in grado di realizzare, in maniera flessibile e semi-automatica, un'ampia gamma di oggetti [17]}
}

\newglossaryentry{NFC}
{
	name=\glslink{NFC}{NFC},
    text=NFC,
    sort=nfc,
    description={acronimo dell'inglese \textit{Near Field Communication}, ossia comunicazione di prossimità, è una tecnologia che fornisce connettività senza fili (RF) bidirezionale a corto raggio (fino a un massimo di 10 cm) [18]}
}

\newglossaryentry{SWOT}
{
	name=\glslink{SWOT}{SWOT, analisi},
    text=analisi SWOT,
    sort=SWOT,
    description={è uno strumento di pianificazione strategica usato per valutare i punti di forza (\textit{Strengths}), le debolezze (\textit{Weaknesses}), le opportunità (\textit{Opportunities}) e le minacce (\textit{Threats}) di un progetto o in un'impresa o in ogni altra situazione in cui un'organizzazione o un individuo debba svolgere una decisione per il raggiungimento di un obiettivo [19]}
}


%\newglossaryentry{}
%{
%	name=\glslink{}{},
%    text=,
%    sort=,
%    description={}
%}
 % database di termini
\makeglossaries


%**************************************************************
% Impostazioni di graphicx
%**************************************************************
\graphicspath{{immagini/}} % cartella dove sono riposte le immagini


%**************************************************************
% Impostazioni di hyperref
%**************************************************************
\hypersetup{
    %hyperfootnotes=false,
    %pdfpagelabels,
    %draft,	% = elimina tutti i link (utile per stampe in bianco e nero)
    colorlinks=true,
    linktocpage=true,
    pdfstartpage=1,
    pdfstartview=FitV,
    % decommenta la riga seguente per avere link in nero (per esempio per la stampa in bianco e nero)
    colorlinks=false, linktocpage=false, pdfborder={0 0 0}, pdfstartpage=1, pdfstartview=FitV,
    breaklinks=true,
    pdfpagemode=UseNone,
    pageanchor=true,
    pdfpagemode=UseOutlines,
    plainpages=false,
    bookmarksnumbered,
    bookmarksopen=true,
    bookmarksopenlevel=1,
    hypertexnames=true,
    pdfhighlight=/O,
    %nesting=true,
    %frenchlinks,
    urlcolor=webbrown,
    linkcolor=RoyalBlue,
    citecolor=webgreen,
    %pagecolor=RoyalBlue,
    %urlcolor=Black, linkcolor=Black, citecolor=Black, %pagecolor=Black,
    pdftitle={\myTitle},
    pdfauthor={\textcopyright\ \myName, \myUni, \myFaculty},
    pdfsubject={},
    pdfkeywords={},
    pdfcreator={pdfLaTeX},
    pdfproducer={LaTeX}
}

%**************************************************************
% Impostazioni di itemize
%**************************************************************
\renewcommand{\labelitemi}{$\bullet$}

%\renewcommand{\labelitemi}{$\bullet$}
%\renewcommand{\labelitemii}{$\cdot$}
%\renewcommand{\labelitemiii}{$\diamond$}
%\renewcommand{\labelitemiv}{$\ast$}


%**************************************************************
% Impostazioni di listings
%**************************************************************
%\lstset{
%    language=[Sharp]C,
%    keywordstyle=\color{RoyalBlue}, %\bfseries,
%    basicstyle=\small\ttfamily,
 %   %identifierstyle=\color{NavyBlue},
%    commentstyle=\color{Green}\ttfamily,
%    stringstyle=\rmfamily,
%    numbers=none, %left,%
%    numberstyle=\scriptsize, %\tiny
%    stepnumber=5,
%    numbersep=8pt,
%    showstringspaces=false,
%    breaklines=true,
%   frameround=ftff,
%    frame=single
%}

\definecolor{background}{RGB}{39, 40, 34}
\definecolor{string}{RGB}{230, 219, 116}
\definecolor{comment}{RGB}{117, 113, 94}
\definecolor{normal}{RGB}{248, 248, 242}
\definecolor{identifier}{RGB}{166, 226, 46}

\lstset{
  language=PHP,                			% choos	e the language of the code
  numbers=left,                   		% where to put the line-numbers
  stepnumber=1,                   		% the step between two line-numbers.        
  numbersep=5pt,                  		% how far the line-numbers are from the code
  numberstyle=\tiny\color{black}\ttfamily,
  backgroundcolor=\color{background},  		% choose the background color. You must add \usepackage{color}
  showspaces=false,               		% show spaces adding particular underscores
  showstringspaces=false,         		% underline spaces within strings
  showtabs=false,                 		% show tabs within strings adding particular underscores
  tabsize=4,                      		% sets default tabsize to 2 spaces
  captionpos=b,                   		% sets the caption-position to bottom
  breaklines=true,                		% sets automatic line breaking
  breakatwhitespace=true,         		% sets if automatic breaks should only happen at whitespace
  title=\lstname,                 		% show the filename of files included with \lstinputlisting;
  basicstyle=\color{normal}\ttfamily,					% sets font style for the code
  keywordstyle=\color{magenta}\ttfamily,	% sets color for keywords
  stringstyle=\color{string}\ttfamily,		% sets color for strings
  commentstyle=\color{comment}\ttfamily,	% sets color for comments
  emph={format_string, eff_ana_bf, permute, eff_ana_btr},
  emphstyle=\color{identifier}\ttfamily
} 



%**************************************************************
% Impostazioni di xcolor
%**************************************************************
\definecolor{webgreen}{rgb}{0,.5,0}
\definecolor{webbrown}{rgb}{.6,0,0}


%**************************************************************
% Altro
%**************************************************************

\newcommand{\omissis}{[\dots\negthinspace]} % produce [...]

% eccezioni all'algoritmo di sillabazione
\hyphenation
{
    ma-cro-istru-zio-ne
    gi-ral-din
}

\newcommand{\sectionname}{sezione}
\addto\captionsitalian{\renewcommand{\figurename}{Figura}
                       \renewcommand{\tablename}{Tabella}}

\newcommand{\glsfirstoccur}{\ap{{[g]}}}

\newcommand{\intro}[1]{\emph{\textsf{#1}}}

%**************************************************************
% Environment per ``rischi''
%**************************************************************
\newcounter{riskcounter}                % define a counter
\setcounter{riskcounter}{0}             % set the counter to some initial value

%%%% Parameters
% #1: Title
\newenvironment{risk}[1]{
    \refstepcounter{riskcounter}        % increment counter
    \par \noindent                      % start new paragraph
    \textbf{\arabic{riskcounter}. #1}   % display the title before the 
                                        % content of the environment is displayed 
}{
    \par\medskip
}

\newcommand{\riskname}{Rischio}

\newcommand{\riskdescription}[1]{\textbf{\\Descrizione:} #1.}

\newcommand{\risksolution}[1]{\textbf{\\Soluzione:} #1.}

%**************************************************************
% Environment per ``use case''
%**************************************************************
\newcounter{usecasecounter}             % define a counter
\setcounter{usecasecounter}{0}          % set the counter to some initial value

%%%% Parameters
% #1: ID
% #2: Nome
\newenvironment{usecase}[2]{
    \renewcommand{\theusecasecounter}{\usecasename #1}  % this is where the display of 
                                                        % the counter is overwritten/modified
    \refstepcounter{usecasecounter}             % increment counter
    \vspace{10pt}
    \par \noindent                              % start new paragraph
    {\large \textbf{\usecasename #1: #2}}       % display the title before the 
                                                % content of the environment is displayed 
    \medskip
}{
    \medskip
}

\newcommand{\usecasename}{UC}

\newcommand{\usecaseactors}[1]{\textbf{\\Attori Principali:} #1. \vspace{4pt}}
\newcommand{\usecasepre}[1]{\textbf{\\Precondizioni:} #1. \vspace{4pt}}
\newcommand{\usecasedesc}[1]{\textbf{\\Descrizione:} #1. \vspace{4pt}}
\newcommand{\usecasepost}[1]{\textbf{\\Postcondizioni:} #1. \vspace{4pt}}
\newcommand{\usecasealt}[1]{\textbf{\\Scenario Alternativo:} #1. \vspace{4pt}}

%**************************************************************
% Environment per ``namespace description''
%**************************************************************

\newenvironment{namespacedesc}{
    \vspace{10pt}
    \par \noindent                              % start new paragraph
    \begin{description} 
}{
    \end{description}
    \medskip
}

\newcommand{\classdesc}[2]{\item[\textbf{#1:}] #2}